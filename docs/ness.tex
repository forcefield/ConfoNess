\documentclass[aps,groupedaddress]{revtex4}
\usepackage{amsmath}

\newcommand*{\mat}[1]{\mathbf{#1}}
\newcommand*{\nullspace}[1]{\mathrm{null}(#1)}
\newcommand*{\rank}[1]{\mathrm{rank}({#1})}
\newcommand*{\diag}[1]{\mathrm{diag}({#1})}

\begin{document}

\title{Nonequilibrium steady state analysis}

\author{Huafeng Xu}

\maketitle

\section{Steady state equations}

Consider the following set of $R$ chemical reactions
\begin{equation}
\sum_{i=1}^{M} c_{ki} m_i = 0  \text{ for } k=1,2,\dots,R
\label{eqn:set-of-reactions}
\end{equation}

In the reactions above, $c_{ki} > 0$ indicates that molecule $m_i$ is
a reactant--and $c_{ki} < 0$ indicates that molecule $m_i$ is a
product--of reaction $k$.  The forward and reverse rate constants for
reaction $k$ will be denoted as $f_k$ and $r_k$.

The set of chemical reactions can thus be represented by a $R\times M$
(often sparse) matrix $\mat{C} = \{ c_{ki} \}$ and the associated
rates $\{ (f_k, r_k), k=1,2,\dots,R \}$.

The reactive flux for the reaction $k$ is given by
\begin{equation}
J_k = f_k \prod_{j|c_{kj}>0} m_j^{c_{kj}} - r_k \prod_{j|c_{kj}<0} m_j^{-c_{kj}}
\label{eqn:reaction-flux}
\end{equation}

And the flux of all the reactions can be represented by the vector 
$\vec{J} = (J_1, J_2, \dots, J_R)^t$.

The rate of change of the molecule $i$ is
\begin{equation}
\dot{m}_i = - \sum_{k=1}^R c_{ki} J_k 
\end{equation}
%
or in matrix form
\begin{equation}
\dot{\vec{m}} = - \mat{C}^t\cdot \vec{J}
\label{eqn:rate-of-change}
\end{equation}

Let $\mat{N} = \nullspace{\mat{C}}$ be the $M\times (M-\rank{\mat{C}})$
matrix representing the null space of $\mat{C}$.  We have
\begin{equation}
\mat{N}^t \dot{\vec{m}} = -(\mat{C} \mat{N})^t \vec{J} = 0
\end{equation}
{\it i.e.},
\begin{equation}
\mat{N}^t \vec{m} = \mat{N}^t \vec{m}(t=0)
\label{eqn:mass-conservation}
\end{equation}
is constant.  Eq.~\ref{eqn:mass-conservation} reflects the mass
conservation.

We are interested in the steady state, 
\begin{equation}
\mat{C}^t\cdot \vec{J} = -\dot{\vec{m}} = \vec{0}
\label{eqn:steady-state}
\end{equation}

To reduce Eq.~\ref{eqn:steady-state} into linearly independent
equations, consider the singular value decomposition $\mat{C} =
\mat{U}\mat{\Sigma}\mat{V}^t$, where $\mat{\Sigma} = ( \diag{\sigma_1,
  \sigma_2, \dots, \sigma_{\rank{\mat{C}}}}, \mat{0}_{R\times(M -
  \rank{\mat{C}})} )$ is the diagonal matrix of the singular
values. We will denote $\mat{\Sigma}_s = \diag{\sigma_1, \sigma_2,
  \dots, \sigma_{\rank{\mat{C}}}}$ to be the
$\rank{\mat{C}}\times\rank{\mat{C}}$ diagonal submatrix of
$\mat{\Sigma}$.

To derive the linearly independent equations for the steady state, let
$\mat{U}_s = ( \vec{U}_1, \vec{U}_2, \dots, \vec{U}_{\rank{\mat{C}}} )$ be the
$R\times\rank{\mat{C}}$ matrix whose column vectors $\{
\vec{U}_{i=1,2,\dots, \rank{\mat{C}}} \}$ span the range of $\mat{C}$. From
Eq. 7, we have
\begin{eqnarray}
\vec{0} &=& \mat{C}^t\cdot\vec{J} 
\nonumber \\
&=& \mat{V} \Sigma^t \mat{U}^t\cdot\vec{J}
\nonumber \\
&=& \mat{V} \left( \begin{array}{cc}
  \mat{\Sigma}_s & \mat{0}_{\rank{\mat{C}}\times(R - \rank{\mat{C}})} \\
  \mat{0}_{(M-\rank{\mat{C}})\times\rank{\mat{C}}} & \mat{0}_{(M-\rank{\mat{C}})\times(R - \rank{\mat{C}})} 
\end{array} \right) 
\left( 
\begin{array}{c}
\mat{U}_s^t \\
\vec{U}_{\rank{\mat{C}} + 1}^t \\
\vdots \\
\vec{U}_R^t 
\end{array} \right) \cdot\vec{J}
\nonumber \\
&=& \mat{V} \left(
\begin{array}{c}
\mat{\Sigma}_s \mat{U}_s^t\cdot\vec{J}\\
\mat{0}_{(M-\rank{\mat{C}})\times R}
\end{array} \right)
\label{eqn:independent-ness-derivation}
\end{eqnarray}
   
Because both $\mat{V}$ and $\mat{\Sigma}_s$ are full-rank, 
invertible matrices, Eq.~\ref{eqn:independent-ness-derivation}
implies
\begin{equation}
\mat{U}_s^t\cdot \vec{J} = 0
\label{eqn:independent-ness}
\end{equation}

Eq.~\ref{eqn:independent-ness} represents $\rank{\mat{C}}$ independent
equations, together with Eq.~\ref{eqn:mass-conservation}, which has
$M-\rank{\mat{C}}$ independent equations, they make up $M$ independent
equations for the steady state concentrations of the $M$ molecules.

If $\rank{\mat{C}} = R$, $\mat{U}_s^t$ becomes a $R\times R$ square invertible
matrix, and Eq.~\ref{eqn:independent-ness} implies
\begin{equation}
\vec{J} = \vec{0}
\label{eqn:detailed-balance}
\end{equation}
which is the condition of detailed balance: the flux of each
individual reaction is zero at the steady state.

Solving the kinetics by Eq.~\ref{eqn:rate-of-change} or the steady state
by Eq.~\ref{eqn:independent-ness} and Eq.~\ref{eqn:mass-conservation}
require the computation of their derivatives with respect to the
concentrations $\vec{m}$.  The derivative of $J_k$ w.r.t. $m_i$ is given by
\begin{equation}
\partial_i J_k \equiv \frac{\partial J_k}{\partial m_i}
= f_k c_{ki}^{+} m_i^{-1} \prod_{j|c_{kj}>0} m_j^{c_{kj}} + r_k c_{ki}^{-} m_i^{-1} \prod_{j|c_{kj}<0} m_j^{-c_{kj}}
\end{equation}
where $c_{ki}^{+} > 0$ is the coefficient of $m_i$ on the reactant
side of reaction $k$, and $c_{ki}^{-} < 0$ is its coefficient on the
product side.  Note that the same molecule may appear on both sides of the
same reaction.  These partial derivatives form the $R\times M$ Jacobian matrix 
\begin{equation}
(\partial_{\vec{m}} \vec{J})_{ki} = \{ \partial_i J_k \}
\end{equation}

The derivatives of Eq.~\ref{eqn:mass-conservation} with respect to $m_i$ is 
staightforward:
\begin{equation}
\partial_{\vec{m}} \left(N^t \vec{m}\right) = \mat{N}^t \mat{I} = \mat{N}^t
\end{equation}

Often, we need to know how the steady state changes with respect to
the rate constants $f_k$ and $r_k$, {\it e.g.}, when we want to fit
the rate constants to the experimentally measured concentrations at
steady states.  To obtain the derivative of the steady state
concentrations w.r.t. a kinetic parameter $\theta$, we differentiate
Eq.~\ref{eqn:mass-conservation} and Eq.~\ref{eqn:independent-ness}:
\begin{eqnarray}
\vec{0} &=& \mat{N}^t d_\theta \vec{m} 
\nonumber \\
\vec{0} &=& \mat{U}_s^t \cdot d_\theta \vec{J} = \mat{U}_s^t \cdot \left(
  \partial_\theta \vec{J} + \partial_{\vec{m}}{\vec{J}} \cdot d_\theta \vec{m} \right) 
\end{eqnarray}
which yields the linear system of equations for $d_\theta \vec{m}$
\begin{equation}
\left( \begin{array}{c}
\mat{N}^t \\
\mat{U}_s^t\cdot \partial_{\vec{m}} \vec{J}
\end{array} \right)\cdot d_\theta\vec{m} = 
\left( \begin{array}{c}
\vec{0}_{M-\rank{\mat{C}}} \\
-\mat{U}_s^t\cdot \partial_\theta \vec{J}
\end{array} \right)
\label{eqn:ness-param-derivative}
\end{equation}

The partial derivatives of $J_{k'}$ with respect to $\theta = f_{k}$
and $\theta = r_{k}$ are
\begin{eqnarray}
\partial_{f_{k}} J_{k'} &=& \delta_{kk'} \prod_{j|c_{kj}>0} m_j^{c_{kj}}
\nonumber \\
\partial_{r_{k}} J_{k'} &=& -\delta_{kk'} \prod_{j|c_{kj}<0} m_j^{-c_{kj}}
\end{eqnarray}
or in vector form:
\begin{eqnarray}
\partial_{f_{k}} \vec{J} &=& \prod_{j|c_{kj}>0} m_j^{c_{kj}} \vec{e}_k
\nonumber \\
\partial_{r_{k}} \vec{J} &=& -\prod_{j|c_{kj}<0} m_j^{-c_{kj}} \vec{e}_k
\end{eqnarray}
where $\vec{e}_k$ is the $k$'th column vector of the $R\times R$
identity matrix.

\section{Numerical considerations}

The reaction rate constants--and the steady state concentrations--in a
set of reactions sometimes span a few orders of magnitude.  This can
lead to numerical instabilities in the solutions of
Eq.~\ref{eqn:rate-of-change} and Eq.~\ref{eqn:steady-state}.  To ameliorate 
the numerical problems, we scale the concentrations by
%
\begin{equation}
m_i' = \alpha_i^{-1} m_i \text{ for }i=1,2,\dots,M
\end{equation}

The rate of change for $\vec{m}'$ is then
\begin{equation}
\dot{m}_i' = \alpha_i^{-1} \dot{m}_i = \alpha_i^{-1}\sum_{k=1}^R c_{ki}\left(f_k\prod_{j|c_{kj}>0} \alpha_j^{c_{kj}} \dot{m}_j'^{c_{kj}} - r_k\prod_{j|c_{kj}<0}\alpha_j^{-c_{kj}} \dot{m}_j^{-c_{kj}}\right)
\label{eqn:scaled-rate-of-change}
\end{equation}

If $\{ m_i' \}$ are all of approximately the same order-of-magnitude, the numerical errors will be due to the disparaty in the scaled rate-of-constants
\begin{equation}
f_k' = f_k \prod_{j|c_{kj}>0} \alpha_j^{c_{kj}}
\label{eqn:scaled-forward-rate}
\end{equation}
and
\begin{equation}
r_k' = r_k \prod_{j|c_{kj}<0} \alpha_j^{-c_{kj}}
\label{eqn:scaled-reverse-rate}
\end{equation}
If they span different orders-of-magnitude, the numerical precision
will be poor in computing differences between the forward and reverse
fluxes and in computing the sum of the net fluxes across different
reactions in Eq.~\ref{eqn:scaled-rate-of-change}.  

We thus look for $\vec{\alpha} = (\alpha_1, \alpha_2, \dots,
\alpha_M)^t$ that make all $f_k'$ and $r_k'$ of order 1, {\it i.e.}  
\begin{eqnarray}
\sum_{j|c_{kj}>0} c_{kj}\ln\alpha_j &=& -\ln f_k
\nonumber \\
\sum_{j|c_{kj}<0} c_{kj}\ln\alpha_j &=& \ln r_k
\label{eqn:scaling-factors}
\end{eqnarray}
Or in matrix form
\begin{equation}
\left( \begin{array}{c}
 \mat{C}^+ \\
 \mat{C}^- \end{array} \right) \ln\vec{\alpha}
= \left( \begin{array}{c}
  -\ln \vec{f} \\
  \ln \vec{r} \end{array} \right)
\label{eqn:scaling-factors-matrix-equation}
\end{equation}
where $\mat{C}^+$ is the matrix of $\mat{C}$ with the negative
elements set to zero, and $\mat{C}^-$ is the matrix of $\mat{C}$ with
the positive elements set to zero.
Eq.~\ref{eqn:scaling-factors-matrix-equation} can be solved by
the standard least square solver.

The equation for mass conservation (Eq.~\ref{eqn:mass-conservation})
for the scaled concentration becomes
\begin{equation}
\mat{N}^t \diag{\vec{\alpha}} \vec{m}' = \mat{N}^t \vec{m} = \mat{N}^t\vec{m}(t=0)
\label{eqn:scaled-mass-conservation}
\end{equation}

The kinetic equation (Eq.~\ref{eqn:rate-of-change}) becomes 
\begin{equation}
\dot{\vec{m}}' = -\diag{\alpha}^{-1} \mat{C}^t\cdot \vec{J}'
\label{eqn:scaled-kinetics}
\end{equation}

So far, there is no evidence that this numerical stabilization makes
any difference.
\end{document}
